\documentclass{article}

\usepackage{amsmath}
\usepackage{amssymb}
\usepackage{graphicx}

\begin{document}

\title{Different Ways of Calculating Pi}

\author{Badger Code}

\maketitle

\section{Introduction}

Pi is a mathematical constant that is the ratio of the circumference of a circle to its diameter. It is an irrational number, which means that it cannot be expressed as a fraction of two integers. Pi has been calculated to trillions of digits, but its exact value is unknown.

There are many different ways to calculate pi. Some of these methods are ancient, while others are more recent. Some methods are based on geometric principles, while others are based on mathematical series.

In this paper, we will discuss some of the different ways of calculating pi. We will start with some of the most basic methods and then move on to more advanced methods. We will also discuss the accuracy of each method and the computational resources required.

\section{Basic Methods}

One of the simplest ways to calculate pi is to use the following formula:

\begin{equation}
\pi = \frac{2 \cdot \sqrt{3}}{d}
\end{equation}

where $d$ is the diameter of the circle. This formula can be derived from the Pythagorean Theorem.

Another simple way to calculate pi is to use the following formula:

\begin{equation}
\pi = \frac{4}{3} \cdot \frac{A}{r^2}
\end{equation}

where $A$ is the area of the circle and $r$ is the radius of the circle. This formula can be derived from the formula for the area of a circle.

These two methods are very simple, but they are not very accurate. They can only be used to calculate pi to a few decimal places.

\section{Advanced Methods}

There are many more advanced methods for calculating pi. Some of these methods are based on geometric principles, while others are based on mathematical series.

One of the most famous geometric methods for calculating pi is the Archimedes method. This method involves inscribing and circumscribing regular polygons in a circle. The ratio of the perimeter of the inscribed polygon to the perimeter of the circumscribed polygon approaches pi as the number of sides of the polygon increases.

Another geometric method for calculating pi is the Monte Carlo method. This method involves randomly generating points in a square that is inscribed in a circle. The ratio of the number of points that fall inside the circle to the total number of points approaches pi as the number of points increases.

There are also many mathematical series that can be used to calculate pi. One of the most famous series is the Leibniz series:

\begin{equation}
\pi = 4 \cdot \sum_{n = 0}^{\infty} \frac{(-1)^n}{2n + 1}
\end{equation}

This series converges very slowly, but it can be used to calculate pi to a high degree of accuracy.

Another mathematical series for calculating pi is the Machin series:

\begin{equation}
\pi = 16 \cdot \left( \frac{1}{4} - \frac{1}{3} + \frac{1}{5} - \frac{1}{7} + \frac{1}{9} - \cdots \right)
\end{equation}

This series converges much faster than the Leibniz series, and it can be used to calculate pi to a very high degree of accuracy.

\section{Accuracy and Computational Resources}

The accuracy of the different methods for calculating pi depends on the method itself and the number of digits that are calculated. The geometric methods are generally more accurate than the mathematical series.

The computational resources required for calculating pi also depends on the method itself and the number of digits that are calculated. The geometric methods are generally more computationally expensive than the mathematical series.

\subsection{Calculating Pi Using 2 Squares' Collisions}

One interesting way to calculate pi is to use the collisions of two squares. This method was first proposed by Gregory Galperin in 2003.

The idea is to consider two squares, one inside the other. The smaller square is inscribed in a circle, and the larger square is circumscribed around the circle. The two squares are then allowed to collide.

As the squares collide, they will bounce off of each other. The number of times that the squares bounce off of each other is proportional to pi.

To calculate pi using this method, we can do the following:

1. Create two squares, one inside the other.
2. Allow the squares to collide.
3. Count the number of times that the squares bounce off of each other.
4. Multiply the number of bounces by a constant factor.

The constant factor is chosen so that the calculated value of pi is close to the actual value of pi.

This method is a very simple way to calculate pi. It is also a very accurate method. The calculated value of pi will be close to the actual value of pi, even with a small number of bounces.

\section{Conclusion}

There are many different ways of calculating pi. Some of these methods are simple, while others are more advanced. Some methods are accurate, while others are not. Some methods are computationally expensive, while others are not.

The best method for calculating pi depends on the specific needs of the application. If a high degree of accuracy is required, then a geometric method or a mathematical series with a large number of terms should be used. If computational resources are limited, then a simpler method with a lower degree of accuracy should be used.
\end{document}