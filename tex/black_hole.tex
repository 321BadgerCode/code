\documentclass{article}
\usepackage{graphicx}
\usepackage{amsmath}
\usepackage{amssymb}
\usepackage{lipsum} % For dummy text, remove this in the final version

\title{Black Holes: Properties, Formation, and Astrophysical Significance}
\author{Badger Code}
\date{\today}

\begin{document}

\maketitle

\section{Introduction}
\label{sec:introduction}
% Provide an overview of black holes, their historical context, and their significance in modern astrophysics.

\section{Theoretical Foundation}
\label{sec:theoretical_foundation}
% Discuss the theory of general relativity, Einstein's field equations, and the Schwarzschild solution.

\subsection{Schwarzschild Metric}
\label{subsec:schwarzschild_metric}
% Explain the Schwarzschild metric and its significance in describing non-rotating black holes.

% Visualize the Schwarzschild metric using a 2D plot of the spacetime curvature.

\subsection{Black Hole Properties}
\label{subsec:black_hole_properties}
% Describe the key properties of black holes, including event horizon, singularity, and no-hair theorem.

% Create a diagram showcasing the event horizon and singularity of a black hole.

\section{Formation of Black Holes}
\label{sec:formation}
% Discuss the formation process of black holes, including stellar collapse and other mechanisms.

% Include a flowchart illustrating the different pathways for black hole formation.

\section{Observational Evidence}
\label{sec:observational_evidence}
% Present observational evidence for the existence of black holes, including X-ray binaries and gravitational waves.

% Insert a diagram of an X-ray binary system and a plot of a gravitational wave signal.

\section{Astrophysical Significance}
\label{sec:astrophysical_significance}
% Explore the astrophysical implications of black holes, such as their role in galaxy formation and evolution.

\subsection{Supermassive Black Holes}
\label{subsec:supermassive_bh}
% Discuss supermassive black holes, their presence in the centers of galaxies, and their impact on galactic dynamics.

% Include a visual representation of a galaxy with a supermassive black hole at its center.

\subsection{Black Hole Accretion}
\label{subsec:black_hole_accretion}
% Explain the process of accretion onto black holes and its observable effects.

% Add a diagram showing matter accretion onto a black hole.

\section{Hawking Radiation}
\label{sec:hawking_radiation}
% Introduce Hawking radiation, its theoretical basis, and its implications for black hole thermodynamics.

% Use a plot to illustrate the temperature and evaporation of a black hole over time.

\section{Conclusion}
\label{sec:conclusion}
% Summarize the key points of the paper and discuss future research directions.

\end{document}