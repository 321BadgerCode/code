\documentclass{article}
\usepackage{graphicx}
\usepackage{tikz}
\usepackage{amsmath}
\usepackage{amssymb}
\usetikzlibrary{calc, matrix}

\title{Linear Algebra and Computer Graphics: \\
       The Role of Linear Algebra in Transformations, 3D Rendering, and Image Processing}
\author{Badger Code}
\date{\today}

\begin{document}
\maketitle

\section{Introduction}
Computer graphics is a vibrant and integral field in computer science, encompassing the creation, manipulation, and rendering of visual imagery. Linear algebra plays a fundamental role in computer graphics, providing the mathematical tools to represent and transform 2D and 3D objects, perform 3D rendering, and process images. This paper examines the key applications of linear algebra in computer graphics and how it enables the generation of realistic images and visual effects.

\section{Representing Points and Vectors in Space}
In computer graphics, points and vectors in 2D and 3D space are commonly represented using vectors and matrices.

\subsection{Vector Representation}
A vector in 2D space can be represented as $\mathbf{v} = \begin{bmatrix} x \\ y \end{bmatrix}$, where $x$ and $y$ are the coordinates of the vector. In 3D space, a vector is represented as $\mathbf{v} = \begin{bmatrix} x \\ y \\ z \end{bmatrix}$.

\subsection{Matrix Representation}
Matrices are used to represent transformations, such as translation, rotation, and scaling. For example, a 2D translation matrix can be represented as:
\[ \begin{bmatrix} 1 & 0 & t_x \\ 0 & 1 & t_y \\ 0 & 0 & 1 \end{bmatrix} \]
where $t_x$ and $t_y$ are the translation distances in the $x$ and $y$ directions, respectively.

\section{Transformations in Computer Graphics}
Transformations play a crucial role in computer graphics, enabling the manipulation and positioning of objects in 2D and 3D space.

\subsection{Translation}
Translation involves moving an object from one position to another in the coordinate system. It is achieved by adding a translation vector to the original coordinates of the object.

\subsection{Rotation}
Rotation changes the orientation of an object around a specified point or axis. It is achieved by multiplying the original coordinates of the object by a rotation matrix.

\subsection{Scaling}
Scaling changes the size of an object along the $x$, $y$, and $z$ axes. It is achieved by multiplying the original coordinates of the object by a scaling matrix.

\section{3D Rendering and Projection}
3D rendering is the process of converting 3D scene data into a 2D image. Linear algebra is extensively used in 3D rendering to perform perspective projection and view transformations.

\subsection{Perspective Projection}
Perspective projection is used to represent a 3D scene on a 2D screen, simulating the way the human eye perceives depth.

\begin{center}
\begin{tikzpicture}[scale=0.8]
    \draw[->] (0,0,0) -- (5,0,0) node[below] {$x$};
    \draw[->] (0,0,0) -- (0,5,0) node[left] {$y$};
    \draw[->] (0,0,0) -- (0,0,5) node[below] {$z$};

    \draw (0,0,2) -- (3,0,0) -- (0,3,0) -- cycle;
    \draw (0,0,0) -- (3,0,0) -- (0,3,0) -- cycle;
    \draw (3,0,0) -- (3,0,2) -- (0,3,0);

    \node[draw=none] at (1.5,1.2) {$O$};
    \node[draw=none] at (1.5,-0.2) {$P$};
    \node[draw=none] at (-0.2,1.5) {$P'$};
    \node[draw=none] at (3.3,0.3) {$F$};
\end{tikzpicture}
\end{center}

In this example, $O$ represents the viewer's eye (camera position), $F$ is the focal point, $P$ is a 3D point in the scene, and $P'$ is the corresponding 2D projected point on the screen.

\subsection{View Transformations}
View transformations involve changing the camera's orientation and position to determine the viewer's perspective in the 3D scene.

\section{Image Processing}
Linear algebra is also used in image processing, allowing the manipulation of digital images to enhance, modify, or extract specific features.

\subsection{Image Filtering}
Image filtering is a common image processing technique used to enhance or modify an image. Filters, such as blurring or edge detection filters, are applied to the image using convolution with a kernel matrix.

\subsection{Image Compression}
Linear algebra techniques, such as Singular Value Decomposition (SVD), are used in image compression algorithms to reduce the size of digital images without significant loss of quality.

\subsection{Geometric Image Transformations}
Geometric transformations, such as image rotation, scaling, and translation, can be performed using linear algebra transformations.

\section{Conclusion}
Linear algebra is a foundational mathematical tool in computer graphics and image processing. It enables the representation and manipulation of points and vectors in 2D and 3D space, facilitating transformations, 3D rendering, and image processing. The application of linear algebra in computer graphics allows the creation of visually stunning and realistic images and visual effects, making it an indispensable component of modern computer graphics algorithms and applications.

\end{document}